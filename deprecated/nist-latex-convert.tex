\documentclass{article}

\begin{document}

NIST Special Publication 1500-8


NIST Big Data Interoperability 
Framework:
Volume 8, Reference Architecture 
Interface


NIST Big Data Public Working Group
Reference Architecture Subgroup


Version 0.1
October 19, 2016
http://dx.doi.org/10.6028/NIST.SP.1500-8 


NIST Special Publication 1500-6
Information Technology Laboratory

NIST Big Data Interoperability Framework:
Volume 8, Reference Architecture Interface

Version 0.1

NIST Big Data Public Working Group (NBD-PWG)
Reference Architecture Subgroup
National Institute of Standards and Technology
Gaithersburg, MD 20899


http://dx.doi.org/10.6028/NIST.SP.1500-8 


October 2016

 
U. S. Department of Commerce
Penny Pritzker, Secretary

National Institute of Standards and Technology
Willie May, Under Secretary of Commerce for Standards and Technology and Director


National Institute of Standards and Technology (NIST) Special Publication 1500-8
20 pages (October 19, 2016)

NIST Special Publication series 1500 is intended to capture external perspectives related to NIST 
standards, measurement, and testing-related efforts. These external perspectives can come from 
industry, academia, government, and others. These reports are intended to document external 
perspectives and do not represent official NIST positions.

Certain commercial entities, equipment, or materials may be identified in this document in order to describe an 
experimental procedure or concept adequately. Such identification is not intended to imply recommendation or 
endorsement by NIST, nor is it intended to imply that the entities, materials, or equipment are necessarily the best 
available for the purpose. 

There may be references in this publication to other publications currently under development by NIST in 
accordance with its assigned statutory responsibilities. The information in this publication, including concepts and 
methodologies, may be used by federal agencies even before the completion of such companion publications. Thus, 
until each publication is completed, current requirements, guidelines, and procedures, where they exist, remain 
operative. For planning and transition purposes, federal agencies may wish to closely follow the development of 
these new publications by NIST. 

Organizations are encouraged to review all draft publications during public comment periods and provide feedback 
to NIST. All NIST publications are available at http://www.nist.gov/publication-portal.cfm.


Comments on this publication may be submitted to Wo Chang

National Institute of Standards and Technology
Attn: Wo Chang, Information Technology Laboratory
100 Bureau Drive (Mail Stop 8900) Gaithersburg, MD 20899-8930
Email: SP1500comments@nist.gov 

 
Reports on Computer Systems Technology
The Information Technology Laboratory (ITL) at NIST promotes the U.S. economy and public welfare by 
providing technical leadership for the Nation’s measurement and standards infrastructure. ITL develops 
tests, test methods, reference data, proof of concept implementations, and technical analyses to advance 
the development and productive use of information technology (IT). ITL’s responsibilities include the 
development of management, administrative, technical, and physical standards and guidelines for the 
cost-effective security and privacy of other than national security-related information in federal 
information systems. This document reports on ITL’s research, guidance, and outreach efforts in IT and 
its collaborative activities with industry, government, and academic organizations.

\section*{Abstract}

Big Data is a term used to describe the large amount of data in the networked, digitized, sensor-laden, 
information-driven world. While opportunities exist with Big Data, the data can overwhelm traditional 
technical approaches, and the growth of data is outpacing scientific and technological advances in data 
analytics. To advance progress in Big Data, the NIST Big Data Public Working Group (NBD-PWG) is 
working to develop consensus on important fundamental concepts related to Big Data. The results are 
reported in the NIST Big Data Interoperability Framework series of volumes. This volume, Volume 6, 
summarizes the work performed by the NBD-PWG to characterize Big Data from an architecture 
perspective, presents the NIST Big Data Reference Architecture (NBDRA) conceptual model, and 
discusses the components and fabrics of the NBDRA. 

\section*{Keywords}

Application Provider; Big Data; Big Data characteristics; Data Consumer; Data Provider; Framework 
Provider; Management Fabric; reference architecture; Security and Privacy Fabric; System Orchestrator; 
use cases. 

 
\section*{Acknowledgements}

This document reflects the contributions and discussions by the membership of the NBD-PWG, co-
chaired by Wo Chang of the NIST ITL, Robert Marcus of ET-Strategies, and Chaitanya Baru, University 
of California San Diego Supercomputer Center. 

The document contains input from members of the NBD-PWG: <TBD>
NIST SP1500-8, Version 1 has been collaboratively authored by the NBD-PWG. As of the date of 
publication, there are over <TBD>
NIST acknowledges the specific contributions  to this volume by the following NBD-PWG members:
<TBD>

The editors for this document were <TBD>, and Wo Chang.



\section{Executive Summary (Draft 0.0001)}

This document, NIST Big Data Interoperability Framework: Volume 8, Reference Architecture Interfaces, 
was prepared by the NIST Big Data Public Working Group (NBD-PWG) Reference Architecture 
Subgroup to establish the operational interfaces for management interactions and dataflow with needed 
resources between Reference Architecture components which are defined in the Volume 6 Reference 
Architecture. 

These interfaces, referred to as the NIST Big Data Reference Architecture Interfaces (NBDRAI), was 
crafted by addressing the model (structure and representation) and protocol (execution mechanism) to the 
control of the Big Data application lifecycle under the direction of the System Orchestrator to other 
NBDRA components. Specific interfaces and the interactions between NBDRA components shall be 
described and defined.

The NIST Big Data Interoperability Framework consists of seven volumes, each of which addresses a 
specific key topic, resulting from the work of the NBD-PWG. The seven volumes are:

\begin{itemize}
\item	Volume 1, Definitions
\item	Volume 2, Taxonomies 
\item	Volume 3, Use Cases and General Requirements
\item	Volume 4, Security and Privacy 
\item	Volume 5, Architectures White Paper Survey
\item	Volume 6, Reference Architecture
\item	Volume 7, Standards Roadmap
\end{itemize}


The NIST Big Data Interoperability Framework will be released in three versions, which correspond to 
the three development stages of the NBD-PWG work. The three stages aim to achieve the following with 
respect to the NIST Big Data Reference Architecture (NBDRA).

Stage 1:	 Identify the high-level Big Data reference architecture key components, which are 
technology-, infrastructure-, and vendor-agnostic.
Stage 2:	 Define general interfaces between the NBDRA components.
Stage 3:	 Validate the NBDRA by building Big Data general applications through the general 
interfaces.

Potential areas of future work for the Subgroup during stage 2 are highlighted in Section 1.5 of this 
volume. The current effort documented in this volume reflects concepts developed within the rapidly 
evolving field of Big Data.



\section{INTRODUCTION}

\subsubsection{BACKGROUND}

There is broad agreement among commercial, academic, and government leaders about the remarkable 
potential of Big Data to spark innovation, fuel commerce, and drive progress. Big Data is the common 
term used to describe the deluge of data in today’s networked, digitized, sensor-laden, and information-
driven world. The availability of vast data resources carries the potential to answer questions previously 
out of reach, including the following:

\begin{itemize}
\item	How can a potential pandemic reliably be detected early enough to intervene? 
\item	Can new materials with advanced properties be predicted before these materials have ever been 
synthesized? 
\item	How can the current advantage of the attacker over the defender in guarding against cyber-
security threats be reversed? 
\end{itemize}



There is also broad agreement on the ability of Big Data to overwhelm traditional approaches. The growth 
rates for data volumes, speeds, and complexity are outpacing scientific and technological advances in data 
analytics, management, transport, and data user spheres. 
Despite widespread agreement on the inherent opportunities and current limitations of Big Data, a lack of 
consensus on some important fundamental questions continues to confuse potential users and stymie 
progress. These questions include the following: 

\begin{itemize}
\item	What attributes define Big Data solutions? 
\item	How is Big Data different from traditional data environments and related applications? 
\item	What are the essential characteristics of Big Data environments? 
\item	How do these environments integrate with currently deployed architectures? 
\item	What are the central scientific, technological, and
standardization challenges that need to be  addressed to accelerate the deployment of robust Big Data solutions?
\end{itemize}



Within this context, on March 29, 2012, the White House announced the Big Data Research and 
Development Initiative.  The initiative’s goals include helping to accelerate the pace of discovery in 
science and engineering, strengthening national security, and transforming teaching and learning by 
improving the ability to extract knowledge and insights from large and complex collections of digital 
data.

Six federal departments and their agencies announced more than \$200 million in commitments spread 
across more than 80 projects, which aim to significantly improve the tools and techniques needed to 
access, organize, and draw conclusions from huge volumes of digital data. The initiative also challenged 
industry, research universities, and nonprofits to join with the federal government to make the most of the 
opportunities created by Big Data. 

Motivated by the White House initiative and public suggestions, the National Institute of Standards and 
Technology (NIST) has accepted the challenge to stimulate collaboration among industry professionals to 
further the secure and effective adoption of Big Data. As one result of NIST’s Cloud and Big Data Forum 
held on January 15–17, 2013, there was strong encouragement for NIST to create a public working group 
for the development of a Big Data Interoperability Framework. Forum participants noted that this 
roadmap should define and prioritize Big Data requirements, including interoperability, portability, 
reusability, extensibility, data usage, analytics, and technology infrastructure. In doing so, the roadmap 
would accelerate the adoption of the most secure and effective Big Data techniques and technology.
On June 19, 2013, the NIST Big Data Public Working Group (NBD-PWG) was launched with extensive 
participation by industry, academia, and government from across the nation. The scope of the NBD-PWG 
involves forming a community of interests from all sectors—including industry, academia, and 
government—with the goal of developing consensus on definitions, taxonomies, secure reference 
architectures, security and privacy, andfrom thesea standards roadmap. Such a consensus would 
create a vendor-neutral, technology- and infrastructure-independent framework that would enable Big 
Data stakeholders to identify and use the best analytics tools for their processing and visualization 
requirements on the most suitable computing platform and cluster, while also allowing value-added from 
Big Data service providers.

The NIST Big Data Interoperability Framework consists of seven volumes, each of which addresses a 
specific key topic, resulting from the work of the NBD-PWG. The seven volumes are:

\begin{itemize}
\item	Volume 1, Definitions
\item	Volume 2, Taxonomies 
\item	Volume 3, Use Cases and General Requirements
\item	Volume 4, Security and Privacy 
\item	Volume 5, Architectures White Paper Survey
\item	Volume 6, Reference Architecture
\item	Volume 7, Standards Roadmap
\end{itemize}


The NIST Big Data Interoperability Framework will be released in three versions, which correspond to 
the three stages of the NBD-PWG work. The three stages aim to achieve the following with respect to the 
NIST Big Data Reference Architecture (NBDRA.)
Stage 1:	 Identify the high-level Big Data reference architecture key components, which are 
technology-, infrastructure-, and vendor-agnostic;
Stage 2:	 Define general interfaces between the NBDRA components; and
Stage 3:	 Validate the NBDRA by building Big Data general applications through the general 
interfaces.
Potential areas of future work for the Subgroup during stage 2 are highlighted in Section 1.5 of this 
volume. The current effort documented in this volume reflects concepts developed within the rapidly 
evolving field of Big Data.
\subsubsection{
SCOPE AND OBJECTIVES OF THE REFERENCE ARCHITECTURES }

SUBGROUP
Reference architectures provide “an authoritative source of information about a specific subject area that 
guides and constrains the instantiations of multiple architectures and solutions.”   Reference architectures 
generally serve as a foundation for solution architectures and may also be used for comparison and 
alignment of instantiations of architectures and solutions. 
The goal of the NBD-PWG Reference Architecture Subgroup is to develop an open reference architecture 
for Big Data that achieves the following objectives:

\begin{itemize}
\item	Provides a common language for the various stakeholders;
\item	Encourages adherence to common standards, specifications, and patterns;
\item	Provides consistent methods for implementation of technology to solve similar problem sets;
\item	Illustrates and improves understanding of the various Big Data components, processes, and 
systems, in the context of a vendor- and technology-agnostic Big Data conceptual model; 
\item	Provides a technical reference for U.S. government departments, agencies, and other consumers 
to understand, discuss, categorize, and compare Big Data solutions; and 
\item	Facilitates analysis of candidate standards for interoperability, portability, reusability, and 
extendibility.
\end{itemize}


The NBDRA is a high-level conceptual model crafted to serve as a tool to facilitate open discussion of the 
requirements, design structures, and operations inherent in Big Data. The NBDRA is intended to facilitate 
the understanding of the operational intricacies in Big Data. It does not represent the system architecture 
of a specific Big Data system, but rather is a tool for describing, discussing, and developing system-
specific architectures using a common framework of reference. The model is not tied to any specific 
vendor products, services, or reference implementation, nor does it define prescriptive solutions that 
inhibit innovation. 
The NBDRA does not address the following:

\begin{itemize}
\item	Detailed specifications for any organization’s operational systems;
\item	Detailed specifications of information exchanges or services; and
\item	Recommendations or standards for integration of infrastructure products.
\end{itemize}


\subsubsection{
REPORT PRODUCTION }

A wide spectrum of Big Data architectures have been explored and developed as part of various industry, 
academic, and government initiatives. The development of the NBDRA and material contained in this 
volume involved the following steps:
1.	Announce that the NBD-PWG Reference Architecture Subgroup is open to the public to attract 
and solicit a wide array of subject matter experts and stakeholders in government, industry, and 
academia;
2.	Gather publicly available Big Data architectures and materials representing various stakeholders, 
different data types, and diverse use cases;  
3.	Examine and analyze the Big Data material to better understand existing concepts, usage, goals, 
objectives, characteristics, and key elements of Big Data, and then document the findings using 
NIST’s Big Data taxonomies model (presented in NIST Big Data Interoperability Framework: 
Volume 2, Taxonomies); and
4.	Develop a technology-independent, open reference architecture based on the analysis of Big Data 
material and inputs received from other NBD-PWG subgroups.
\subsubsection{
REPORT STRUCTURE }

The organization of this document roughly corresponds to the process used by the NBD-PWG to develop 
the NBDRA. Following the introductory material presented in Section 1, the remainder of this document 
is organized as follows: 

\begin{itemize}
\item	Section 2 contains high-level, system requirements in support of Big Data relevant to the design 
of the NBDRA and discusses the development of these requirements. 
\item	Section 3 presents the generic, technology-independent NBDRA conceptual model.
\item	Section 4 discusses the five main functional components of the NBDRA.
\item	Section 5 describes the system and life cycle management considerations related to the NBDRA 
management fabric.
\item	Section 6 briefly introduces security and privacy topics related to the security and privacy fabric 
of the NBDRA.
\item	Appendix A summarizes deployment considerations.
\item	Appendix B lists the terms and definitions in this document.
\item	Appendix C provides examples of Big Data logical data architecture options.
\item	Appendix D defines the acronyms used in this document.
\item	Appendix E lists general resources that provide additional information on topics covered in this 
document and specific references in this document.
\end{itemize}



\subsubsection{FUTURE WORK ON THIS VOLUME (NEXT STEP IS VALIDATION)}

This document (Version 1) presents the overall NBDRA components and fabrics with high-level 
description and functionalities. 
Version 2 activities will focus on the definition of general interfaces between the NBDRA components by 
performing the following:

\begin{itemize}
\item	Select use cases from the 62 (51 general and 11 security and privacy) submitted use cases or 
other, to be identified, meaningful use cases;
\item	Work with domain experts to identify workflow and interactions among the NBDRA 
components and fabrics;
\item	Explore and model these interactions within a small-scale, manageable, and well-defined 
confined environment; and 
\item	Aggregate the common data workflow and interactions between NBDRA components and 
fabrics and package them into general interfaces.
Version 3 activities will focus on validation of the NBDRA through the use of the defined NBDRA 
general interfaces to build general Big Data applications. The validation strategy will include the 
following:
\item	Implement the same set of use cases used in Version 2 by using the defined general interfaces;
\item	Identify and implement a few new use cases outside the Version 2 scenarios; and
\item	Enhance general NBDRA interfaces through lessons learned from the implementations in 
Version 3 activities.
\end{itemize}


The general interfaces developed during Version 2 activities will offer a starting point for further 
refinement by any interested parties and is not intended to be a definitive solution to address all 
implementation needs. 
\section{HIGH-LEVEL NBDRA INTERFACE REQUIREMENTS}
The Volume 6 Reference Architecture document provides a list of comprehensive high-level reference 
architecture requirements. To enable interoperability between the NBDRA components, a list of well-
defined NBDRA interface is needed.  
\subsubsection{
SYSTEM ORCHESTRATION REQUIREMENTS}

<Description>
\subsubsection{
APPLICATION PROVIDERS REQUIREMENTS }

<Description>
\subsubsection{
DATA PROVIDERS REQUIREMENTS }

<Description>
\subsubsection{
FRAMEWORK PROVIDERS REQUIREMENTS }

<Description>
\subsubsection{
DATA CONSUMERS REQUIREMENTS }

<Description>
\subsubsection{
SECURITY AND PRIVACY FABRIC REQUIREMENTS }

<Description>
\subsubsection{
SYSTEM MANAGEMENT REQUIREMENTS }

<Description>

\section{NBDRA INTERFACE APPROACH }
<Write up>

\begin{itemize}
\item	Orchestrate via Application Provider to other RA components
o	System Orchestrator (Data Scientists) uses the BD Application Provider as the command 
center to orchestrate dataflow from Data Provider, carryout the BD application lifecycle 
with the help of the BD Framework Provider, and enable Data Consumer to consume Big 
Data processing results 
\item	Agnostic – can plug-in any specific technologies, analytics, IaaSs to support Big Data 
applications at any environments with and without many CPUs/Cores/GPUs/other accelerators:
o	Laptop/desktop
o	Server
o	Data centers
o	clouds
\item	Customizable parameters in two-levels
o	High-level: with defaults parameters
o	Low-level: call out specific parameters and environmental needs
\item	Packaging analytics algorithms/tools (as payload) via standard interface (as transport) to achieve 
interoperability across application domains with the goals for analytics to be
o	Re-usable – available analytics packages for adoption
o	Deployable – customizable analytics tools for deployment
o	Operational – adjustable 
\end{itemize}

 
NBDRA Components
To develop the use cases, publically available information was collected for various Big Data 
architectures in nine broad areas, or application domains. Participants in the NBD-PWG Use Case and 
Requirements Subgroup and other interested parties provided the use case details via a template, which 
helped to standardize the responses and facilitate subsequent analysis and comparison of the use cases. 

Figure 2: NIST Big Data Reference Architecture (NBDRA)

\section{NBDRA FUNCTIONAL INTERFACES}

As outlined in Section 3, the five main functional components of the NBDRA represent the different 
technical roles within a Big Data system. The functional components are listed below and discussed in 
subsequent subsections.

\begin{itemize}
\item	System Orchestrator: Defines and integrates the required data application activities into an 
operational vertical system;
\item	Big Data Application Provider: Executes a data life cycle to meet security and privacy 
requirements as well as System Orchestrator-defined requirements;
\item	Data Provider: Introduces new data or information feeds into the Big Data system;
\item	Big Data Framework Provider: Establishes a computing framework in which to execute 
certain transformation applications while protecting the privacy and integrity of data; and
\item	Data Consumer: Includes end users or other systems that use the results of the Big Data 
Application Provider.
\end{itemize}



\subsubsection{SYSTEM ORCHESTRATOR TO BD APPLICATION PROVIDER }

INTERFACE

The System Orchestrator role includes defining and integrating the required data application activitieS 
Into an operational vertical system. Typically, the System Orchestrator involves a collection of more 
specific roles, performed by one or more actors, which manage and orchestrate the operation of the Big 
Data system. These actors may be human components, software components, or some combination of the 
two. The function of the System Orchestrator is to configure and manage the other components of the Big 
Data architecture to implement one or more workloads that the architecture is designed to execute. The 
workloads managed by the System Orchestrator may be assigning/provisioning framework components to 
individual physical or virtual nodes at the lower level, or providing a graphical user interface that supports 
the specification of workflows linking together multiple applications and components at the higher level. 
The System Orchestrator may also, through the Management Fabric, monitor the workloads and system to 
confirm that specific quality of service requirements are met for each workload, and may actually 
elastically assign and provision additional physical or virtual resources to meet workload requirements 
resulting from changes/surges in the data or number of users/transactions.
\subsubsection{
BD APPLICATION PROVIDER INTERFACE}

The Big Data Application Provider role executes a specific set of operations along the data life cycle to 
meet the requirements established by the System Orchestrator, as well as meeting security and privacy 
requirements. The Big Data Application Provider is the architecture component that encapsulates the 
business logic and functionality to be executed by the architecture. The Big Data Application Provider 
activities include the following:

\begin{itemize}
\item	Collection
\item	Preparation
\item	Analytics
\item	Visualization
\item	Access
\end{itemize}


\subsubsubsection{COLLECTION}

In general, the collection activity of the Big Data Application Provider handles the interface with the Data 
Provider. This may be a general service, such as a file server or web server configured by the System 
Orchestrator to accept or perform specific collections of data, or it may be an application-specific service 
designed to pull data or receive pushes of data from the Data Provider. Since this activity is receiving data 
at a minimum, it must store/buffer the received data until it is persisted through the Big Data Framework 
Provider. This persistence need not be to physical media but may simply be to an in-memory queue or 
other service provided by the processing frameworks of the Big Data Framework Provider. The collection 
activity is likely where the extraction portion of the Extract, Transform, Load (ETL)/Extract, Load, 
Transform (ELT) cycle is performed. At the initial collection stage, sets of data (e.g., data records) of 
similar structure are collected (and combined), resulting in uniform security, policy, and other 
considerations. Initial metadata is created (e.g., subjects with keys are identified) to facilitate subsequent 
aggregation or look-up methods.

\subsubsubsection{PREPARATION}

The preparation activity is where the transformation portion of the ETL/ELT cycle is likely performed, 
although analytics activity will also likely perform advanced parts of the transformation. Tasks performed 
by this activity could include data validation (e.g., checksums/hashes, format checks), cleansing (e.g., 
eliminating bad records/fields), outlier removal, standardization, reformatting, or encapsulating. This 
activity is also where source data will frequently be persisted to archive storage in the Big Data 
Framework Provider and provenance data will be verified or attached/associated. Verification or 
attachment may include optimization of data through manipulations (e.g., deduplication) and indexing to 
optimize the analytics process. This activity may also aggregate data from different Data Providers, 
leveraging metadata keys to create an expanded and enhanced data set.
\subsubsubsection{
ANALYTICS}

The analytics activity of the Big Data Application Provider includes the encoding of the low-level 
business logic of the Big Data system (with higher-level business process logic being encoded by the 
System Orchestrator). The activity implements the techniques to extract knowledge from the data based 
on the requirements of the vertical application. The requirements specify the data processing algorithms 
for processing the data to produce new insights that will address the technical goal. The analytics activity 
will leverage the processing frameworks to implement the associated logic. This typically involves the 
activity providing software that implements the analytic logic to the batch and/or streaming elements of 
the processing framework for execution. The messaging/communication framework of the Big Data 
Framework Provider may be used to pass data or control functions to the application logic running in the 
processing frameworks. The analytic logic may be broken up into multiple modules to be executed by the 
processing frameworks which communicate, through the messaging/communication framework, with 
each other and other functions instantiated by the Big Data Application Provider.
\subsubsubsection{
VISUALIZATION}

The visualization activity of the Big Data Application Provider prepares elements of the processed data 
and the output of the analytic activity for presentation to the Data Consumer. The objective of this activity 
is to format and present data in such a way as to optimally communicate meaning and knowledge. The 
visualization preparation may involve producing a text-based report or rendering the analytic results as 
some form of graphic. The resulting output may be a static visualization and may simply be stored 
through the Big Data Framework Provider for later access. However, the visualization activity frequently 
interacts with the access activity, the analytics activity, and the Big Data Framework Provider (processing 
and platform) to provide interactive visualization of the data to the Data Consumer based on parameters 
provided to the access activity by the Data Consumer. The visualization activity may be completely 
application-implemented, leverage one or more application libraries, or may use specialized visualization 
processing frameworks within the Big Data Framework Provider. 
\subsubsubsection{
ACCESS}

The access activity within the Big Data Application Provider is focused on the communication/interaction 
with the Data Consumer. Similar to the collection activity, the access activity may be a generic service 
such as a web server or application server that is configured by the System Orchestrator to handle specific 
requests from the Data Consumer. This activity would interface with the visualization and analytic 
activities to respond to requests from the Data Consumer (who may be a person) and uses the processing 
and platform frameworks to retrieve data to respond to Data Consumer requests. In addition, the access 
activity confirms that descriptive and administrative metadata and metadata schemes are captured and 
maintained for access by the Data Consumer and as data is transferred to the Data Consumer. The 
interface with the Data Consumer may be synchronous or asynchronous in nature and may use a pull or 
push paradigm for data transfer. 
\subsubsection{
BD APPLICATION PROVIDER TO DATA PROVIDER INTERFACE}

The Data Provider role introduces new data or information feeds into the Big Data system for discovery, 
access, and transformation by the Big Data system. New data feeds are distinct from the data already in 
use by the system and residing in the various system repositories. Similar technologies can be used to 
access both new data feeds and existing data. The Data Provider actors can be anything from a sensor, to 
a human inputting data manually, to another Big Data system.
\subsubsection{BD APPLICATION PROVIDER TO FRAMEWORK PROVIDER
INTERFACE}


The Big Data Framework Provider typically consists of one or more hierarchically organized instances of 
the components in the NBDRA IT value chain (Figure 2). There is no requirement that all instances at a 
given level in the hierarchy be of the same technology. In fact, most Big Data implementations are 
hybrids that combine multiple technology approaches in order to provide flexibility or meet the complete 
range of requirements, which are driven from the Big Data Application Provider. 
\subsubsubsection{
INFRASTRUCTURE FRAMEWORKS}

This Big Data Framework Provider element provides all of the resources necessary to host/run the 
activities of the other components of the Big Data system. Typically, these resources consist of some 
combination of physical resources, which may host/support similar virtual resources. These resources are 
generally classified as follows:
\subsubsubsection{
DATA PLATFORM FRAMEWORKS}

Data Platform Frameworks provide for the logical data organization and distribution combined with the 
associated access application programming interfaces (APIs) or methods. The frameworks may also 

\subsubsubsection{
PROCESSING FRAMEWORKS}

The processing frameworks for Big Data provide the necessary infrastructure software to support 
implementation of applications that can deal with the volume, velocity, variety, and variability of data. 
Processing frameworks define how the computation and processing of the data is organized. Big Data 
applications rely on various platforms and technologies to meet the challenges of scalable data analytics 
and operation. 
\subsubsubsection{
MESSAGING/COMMUNICATIONS FRAMEWORKS}

Messaging and communications frameworks have their roots in the High Performance Computing (HPC) 
environments long popular in the scientific and research communities. Messaging/Communications 
Frameworks were developed to provide APIs for the reliable queuing, transmission, and receipt of data 
\subsubsubsection{
RESOURCE MANAGEMENT FRAMEWORK}

As Big Data systems have evolved and become more complex, and as businesses work to leverage limited 
computation and storage resources to address a broader range of applications and business challenges, the 
requirement to effectively manage those resources has grown significantly. While tools for resource 
management and “elastic computing” have expanded and matured in response to the needs of cloud 
providers and virtualization technologies, Big Data introduces unique requirements for these tools. 
However, Big Data frameworks tend to fall more into a distributed computing paradigm, which presents 
additional challenges. 

\subsubsection{BD APPLICATION PROVIDER TO DATA CONSUMER INTERFACE}

Similar to the Data Provider, the role of Data Consumer within the NBDRA can be an actual end user or 
another system. In many ways, this role is the mirror image of the Data Provider, with the entire Big Data 
framework appearing like a Data Provider to the Data Consumer. The activities associated with the Data 
Consumer role include the following:

\begin{itemize}
\item	Search and Retrieve
\item	Download
\item	Analyze Locally
\item	Reporting
\item	Visualization
\item	Data to Use for Their Own Processes
\end{itemize}


\section{NBDRA SECURITY AND PRIVACY FABRIC }
INTERFACE
<Security and privacy …>

\subsubsection{
SECURITY AND PRIVACY}

The characteristics of Big Data pose system management challenges on traditional management Big Data 
Life cycle Management

\section{NBDRA MANAGEMENT FABRIC INTERFACES}
<Description…>

\subsubsection{SYSTEM MANAGEMENT}

The characteristics of Big Data pose system management challenges on traditional management Big Data 
Life cycle Management
\section{CONCLUSION }
<Description>

\appendix

Appendix D: Acronyms 
ACID 	Atomicity, Consistency, Isolation, Durability
API 	application programming interface 
ASCII 	American Standard Code for Information Interchange
BASE 	Basically Available, Soft state, Eventual consistency

Appendix E: Resources and References
GENERAL RESOURCES
The following resources provide additional information related to Big Data architecture. 
Big Data Public Working Group, “NIST Big Data Program,” National Institute for Standards and 
Technology, June 26, 2013, http://bigdatawg.nist.gov .
Doug Laney, “3D Data Management: Controlling Data Volume, Velocity, and Variety,” Gartner, February 
6, 2001, http://blogs.gartner.com/doug-laney/files/2012/01/ad949-3D-Data-Management-Controlling-
Data-Volume-Velocity-and-Variety.pdf.
DOCUMENT REFERENCES
  “Contributors” are members of the NIST Big Data Public Working Group who dedicated great effort to prepare 
and substantial time on a regular basis to research and development in support of this document.
  Many of the architecture use cases were originally collected by the NBD-PWG Use Case and Requirements 
Subgroup and can be accessed at http://bigdatawg.nist.gov/usecases.php.
  The White House Office of Science and Technology Policy, “Big Data is a Big Deal,” OSTP Blog, accessed 
February 21, 2014, http://www.whitehouse.gov/blog/2012/03/29/big-data-big-deal.
  Office of the Assistant Secretary of Defense, “Reference Architecture Description,” U.S. Department of Defense, 
June 2010, http://dodcio.defense.gov/Portals/0/Documents/DIEA/Ref_Archi_Description_Final_v1_18Jun10.pdf.
 
 
NIST BIG DATA INTEROPERABILITY FRAMEWORK: VOLUME 8, REFERENCE ARCHITECTURE INTERFACE
vii

7

\end{document}